% This document is part of the emcee3 project.
% Copyright 2015 Dan Foreman-Mackey
%
%  RULES OF THE GAME
%
%  * 80 characters
%  * line breaks at the ends of sentences
%  * eqnarrys ONLY
%

\documentclass[12pt,preprint]{aastex}

\pdfoutput=1

\usepackage{color,hyperref}
\definecolor{linkcolor}{rgb}{0,0,0.5}
\hypersetup{colorlinks=true,linkcolor=linkcolor,citecolor=linkcolor,
            filecolor=linkcolor,urlcolor=linkcolor}
\usepackage{url}
\usepackage{amssymb,amsmath}
\usepackage{subfigure}
\usepackage{booktabs}

\usepackage{natbib}
\bibliographystyle{apj}

% Typography
\newcommand{\project}[1]{\textsl{#1}}
\newcommand{\license}{MIT License}
\newcommand{\paper}{\textsl{Article}}
\newcommand{\foreign}[1]{\emph{#1}}
\newcommand{\etal}{\foreign{et\,al.}}
\newcommand{\etc}{\foreign{etc.}}

\newcommand{\figref}[1]{\ref{fig:#1}}
\newcommand{\Fig}[1]{\figurename~\figref{#1}}
\newcommand{\fig}[1]{\Fig{#1}}
\newcommand{\figlabel}[1]{\label{fig:#1}}
\newcommand{\Tab}[1]{Table~\ref{tab:#1}}
\newcommand{\tab}[1]{\Tab{#1}}
\newcommand{\tablabel}[1]{\label{tab:#1}}
\newcommand{\Eq}[1]{Equation~(\ref{eq:#1})}
\newcommand{\eq}[1]{\Eq{#1}}
\newcommand{\eqalt}[1]{Equation~\ref{eq:#1}}
\newcommand{\eqlabel}[1]{\label{eq:#1}}
\newcommand{\sectionname}{Section}
\newcommand{\Sect}[1]{\sectionname~\ref{sect:#1}}
\newcommand{\sect}[1]{\Sect{#1}}
\newcommand{\sectalt}[1]{\ref{sect:#1}}
\newcommand{\App}[1]{Appendix~\ref{sect:#1}}
\newcommand{\app}[1]{\App{#1}}
\newcommand{\sectlabel}[1]{\label{sect:#1}}

% Algorithms
\usepackage{algorithm}
\usepackage{algorithmicx}
\usepackage[]{algpseudocode}
\newcommand*\Let[2]{\State #1 $\gets$ #2}
\newcommand{\Alg}[1]{Algorithm~\ref{alg:#1}}
\newcommand{\alg}[1]{\Alg{#1}}
\newcommand{\alglabel}[1]{\label{alg:#1}}

% To-do
\newcommand{\todo}[3]{{\color{#2}\emph{#1}: #3}}
\newcommand{\dfmtodo}[1]{\todo{DFM}{red}{#1}}

% Response to referee
\definecolor{mygreen}{rgb}{0, 0.50196, 0}
\newcommand{\response}[1]{#1}
% \newcommand{\response}[1]{{\color{mygreen} {\bf #1}}}

% Notation for this paper.
\newcommand{\T}{{\ensuremath{\mathrm{T}}}}
\newcommand{\bvec}[1]{{\ensuremath{\boldsymbol{#1}}}}
\newcommand{\lnprob}{{\ensuremath{\mathcal{L}}}}
\newcommand{\pos}{{\bvec{q}}}
\newcommand{\mom}{{\bvec{p}}}
\newcommand{\mass}{{\bvec{M}}}
\newcommand{\normal}[2]{{\ensuremath{\mathcal{N}(#1,\,#2)}}}

\begin{document}

\title{%
    Affine-invariant Hamiltonian Monte Carlo
}

\newcommand{\uw}{2}
\newcommand{\sagan}{3}
\author{%
    Daniel~Foreman-Mackey\altaffilmark{1,\uw,\sagan}
}
\altaffiltext{1}         {To whom correspondence should be addressed:
                          \url{danfm@uw.edu}}
\altaffiltext{\uw}       {Astronomy Department, University of Washington,
                          Seattle, WA 98195}
\altaffiltext{\sagan}    {Sagan Fellow}


\begin{abstract}

Hamiltonian Monte Carlo (HMC) sampling is an efficient method for drawing
samples from a probability density when the gradient of the probability with
respect to the parameters can be computed.
We present a simple but effective affine-invariant HMC method that uses an
ensemble of samplers to adaptively update the mass matrix.
We demonstrate the performance of this method on some simple test cases and
compare its computational cost on a real data analysis problem in exoplanet
astronomy.
A well-tested and efficient Python implementation is released alongside this
note.

\end{abstract}

\keywords{%
methods: data analysis
---
methods: statistical
}

\section{Introduction}

Text. \citep{Foreman-Mackey:2013}

Adaptive: \citet{Girolami:2011, Wang:2013, Hoffman:2014}

\section{Hamiltonian Monte Carlo}

\begin{algorithm}
    \caption{Standard implementation of a single HMC step \alglabel{basic-hmc}}
    \begin{algorithmic}
        \Function{HMCStep}{$\lnprob(\pos),\,\pos_t,\,\mass,\,\epsilon,\,L$}
        \State $\mom_t \sim \normal{\bvec{0}}{\mass}$
            \Comment{sample the initial momentum exactly}
        \Let{\pos}{$\pos_t$}
        \State
        \Let{\mom}{$\mom_t + \frac{\epsilon}{2}\,\nabla\lnprob(\pos)$}
            \Comment{run $L$ steps leapfrog integration}
        \For{$l \gets 1 \textrm{ to } L$}
            \Let{\pos}{$\pos + \epsilon\,\mass^{-1}\,\mom$}
            \If{$l < L$}
                \Let{\mom}{$\mom + \epsilon\,\nabla\lnprob(\pos)$}
            \EndIf
        \EndFor
        \Let{\mom}{$\mom + \frac{\epsilon}{2}\,\nabla\lnprob(\pos)$}
            \Comment{synchronize the momentum and position}
        \State
        \State{$r \sim \mathcal{U}(0, 1)$}
        \If{$r < \exp\left[\lnprob(\pos) - \frac{1}{2}\pos^T\mass^{-1}\pos
            - \lnprob(\pos_t)+\frac{1}{2}{\pos_t}^T\mass^{-1}\pos_t \right]$}
            \State\Return{$\pos$}   \Comment{accept}
        \Else
            \State\Return{$\pos_t$} \Comment{reject}
        \EndIf
        \EndFunction
    \end{algorithmic}
\end{algorithm}

\clearpage
\bibliography{emcee-hmc}
\clearpage

\end{document}
