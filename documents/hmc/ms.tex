% This document is part of the emcee3 project.
% Copyright 2015 Dan Foreman-Mackey
%
%  RULES OF THE GAME
%
%  * 80 characters
%  * line breaks at the ends of sentences
%  * eqnarrys ONLY
%

\documentclass[12pt,preprint]{aastex}

\pdfoutput=1

\usepackage{color,hyperref}
\definecolor{linkcolor}{rgb}{0,0,0.5}
\hypersetup{colorlinks=true,linkcolor=linkcolor,citecolor=linkcolor,
            filecolor=linkcolor,urlcolor=linkcolor}
\usepackage{url}
\usepackage{amssymb,amsmath}
\usepackage{subfigure}
\usepackage{booktabs}

\usepackage{natbib}
\bibliographystyle{apj}

% Typography
\newcommand{\project}[1]{\textsl{#1}}
\newcommand{\license}{MIT License}
\newcommand{\paper}{\textsl{Article}}
\newcommand{\foreign}[1]{\emph{#1}}
\newcommand{\etal}{\foreign{et\,al.}}
\newcommand{\etc}{\foreign{etc.}}

\newcommand{\figref}[1]{\ref{fig:#1}}
\newcommand{\Fig}[1]{\figurename~\figref{#1}}
\newcommand{\fig}[1]{\Fig{#1}}
\newcommand{\figlabel}[1]{\label{fig:#1}}
\newcommand{\Tab}[1]{Table~\ref{tab:#1}}
\newcommand{\tab}[1]{\Tab{#1}}
\newcommand{\tablabel}[1]{\label{tab:#1}}
\newcommand{\Eq}[1]{Equation~(\ref{eq:#1})}
\newcommand{\eq}[1]{\Eq{#1}}
\newcommand{\eqalt}[1]{Equation~\ref{eq:#1}}
\newcommand{\eqlabel}[1]{\label{eq:#1}}
\newcommand{\sectionname}{Section}
\newcommand{\Sect}[1]{\sectionname~\ref{sect:#1}}
\newcommand{\sect}[1]{\Sect{#1}}
\newcommand{\sectalt}[1]{\ref{sect:#1}}
\newcommand{\App}[1]{Appendix~\ref{sect:#1}}
\newcommand{\app}[1]{\App{#1}}
\newcommand{\sectlabel}[1]{\label{sect:#1}}

% Algorithms
\usepackage{algorithm}
\usepackage{algorithmicx}
\usepackage[]{algpseudocode}
\newcommand*\Let[2]{\State #1 $\gets$ #2}
\newcommand{\Alg}[1]{Algorithm~\ref{alg:#1}}
\newcommand{\alg}[1]{\Alg{#1}}
\newcommand{\alglabel}[1]{\label{alg:#1}}

% To-do
\newcommand{\todo}[3]{{\color{#2}\emph{#1}: #3}}
\newcommand{\dfmtodo}[1]{\todo{DFM}{red}{#1}}

% Response to referee
\definecolor{mygreen}{rgb}{0, 0.50196, 0}
\newcommand{\response}[1]{#1}
% \newcommand{\response}[1]{{\color{mygreen} {\bf #1}}}

% Notation for this paper.
\newcommand{\T}{{\ensuremath{\mathrm{T}}}}
\newcommand{\bvec}[1]{{\ensuremath{\boldsymbol{#1}}}}
\newcommand{\lnprob}{{\ensuremath{\mathcal{L}}}}
\newcommand{\pos}{{\bvec{q}}}
\newcommand{\mom}{{\bvec{p}}}
\newcommand{\mass}{{\bvec{M}}}
\newcommand{\normal}[2]{{\ensuremath{\mathcal{N}(#1,\,#2)}}}

\begin{document}

\title{%
    Affine-invariant Hamiltonian Monte Carlo
}

\newcommand{\uw}{2}
\newcommand{\sagan}{3}
\author{%
    Daniel~Foreman-Mackey\altaffilmark{1,\uw,\sagan}
}
\altaffiltext{1}         {To whom correspondence should be addressed:
                          \url{danfm@uw.edu}}
\altaffiltext{\uw}       {Astronomy Department, University of Washington,
                          Seattle, WA 98195}
\altaffiltext{\sagan}    {Sagan Fellow}


\begin{abstract}

Hamiltonian Monte Carlo (HMC) sampling is an efficient method for drawing
samples from a probability density when the gradient of the probability with
respect to the parameters can be computed.
We present a simple but effective affine-invariant HMC method that uses an
ensemble of samplers to adaptively update the mass matrix.
We demonstrate the performance of this method on some simple test cases and
compare its computational cost on a real data analysis problem in exoplanet
astronomy.
A well-tested and efficient Python implementation is released alongside this
note.

\end{abstract}

\keywords{%
methods: data analysis
---
methods: statistical
}

\section{Introduction}

% Text. \citep{Foreman-Mackey:2013}
% Adaptive: \citet{Girolami:2011, Wang:2013, Hoffman:2014}
% \section{Hamiltonian Monte Carlo}

Pseudocode for the standard implementation of the Hamiltonian Monte Carlo
(HMC) algorithm \citep{Neal:2011} is shown in \alg{basic-hmc}.
In this implementation, there are $(D^2 + D) / 2 + 2$ tuning parameters, where
$D$ is the dimension of the problem.
Of these parameters, $(D^2 + D) / 2$ are the elements of the positive
definite mass matrix \bvec{M} and the other 2 are the step size $\epsilon$ and
the number of steps $L$.
Most practical applications of HMC fix the mass matrix to a constant (often
set to 1) times the identity and reduce the tuning to only the two parameters
$\epsilon$ and $L$.
Methods have been developed to automatically tune these parameters \citep[for
example][]{Hoffman:2014}.

The major problem with fixing the mass matrix to be diagonal is that the
\emph{units} of the input space can change the performance of the algorithm.
For example, sampling from a Gaussian with different variances in the
different dimensions or covariance between the parameters will be less
efficient than sampling from an isotropic Gaussian.
It has been demonstrated that samplers that satisfy affine invariance can be
very useful for real problems in science where the dynamic range of parameters
can vary by orders of magnitude \citep{Goodman:2010, Foreman-Mackey:2013}.
It turns out that HMC can be simply adapted to an affine-invariant algorithm.

The affine-invariant samplers proposed by \citet{Goodman:2010} sample the
target density by evolving an \emph{ensemble} of parallel MCMC chains (called
``walkers'') where the instantaneous proposal for one walker is conditioned on
the current locations of the other walkers, the \emph{complementary ensemble}.
If the move preserves the conditional distribution of the target walker given
the complementary ensemble, it will also preserve the joint distribution of
the ensemble.
The intuition from these proposals can be incorporated into HMC to derive an
affine-invariant algorithm.




\begin{algorithm}
    \caption{Standard implementation of a single HMC step \alglabel{basic-hmc}}
    \begin{algorithmic}
        \Function{HMCStep}{$\lnprob(\pos),\,\pos_t,\,\mass,\,\epsilon,\,L$}
        \State $\mom_t \sim \normal{\bvec{0}}{\mass}$
            \Comment{sample the initial momentum exactly}
        \Let{\pos}{$\pos_t$}
        \State
        \Let{\mom}{$\mom_t + \frac{\epsilon}{2}\,\nabla\lnprob(\pos)$}
            \Comment{run $L$ steps of leapfrog integration}
        \For{$l \gets 1 \textrm{ to } L$}
            \Let{\pos}{$\pos + \epsilon\,\mass^{-1}\,\mom$}
            \If{$l < L$}
                \Let{\mom}{$\mom + \epsilon\,\nabla\lnprob(\pos)$}
            \EndIf
        \EndFor
        \Let{\mom}{$\mom + \frac{\epsilon}{2}\,\nabla\lnprob(\pos)$}
            \Comment{synchronize the momentum and position}
        \State
        \State{$r \sim \mathcal{U}(0, 1)$}
        \If{$r < \exp\left[\lnprob(\pos) - \frac{1}{2}\mom^T\mass^{-1}\mom
            - \lnprob(\pos_t)+\frac{1}{2}{\mom_t}^T\mass^{-1}\mom_t \right]$}
            \State\Return{$\pos$}   \Comment{accept}
        \Else
            \State\Return{$\pos_t$} \Comment{reject}
        \EndIf
        \EndFunction
    \end{algorithmic}
\end{algorithm}


\begin{algorithm}
    \caption{Affine-invariant HMC \alglabel{ai-hmc}}
    \begin{algorithmic}
\Function{AIHMCStep}{$\lnprob(\pos),\,\{\pos_k\}_{k=1}^K,\,\epsilon,\,L$}
\For{$k \gets 1 \textrm{ to } K$}
    \Let{${\mass_k}^{-1}$}{$\mathrm{Cov}(\pos_{[k]})$}
        \Comment{estimate the empirical mass matrix}
    \State $\mom_k \sim \normal{\bvec{0}}{\mass_k}$
        \Comment{sample the initial momentum exactly}
    \State
    \Let{$\mom^\prime$}{$\mom_k$} \Comment{save the initial coordinates}
    \Let{$\pos^\prime$}{$\pos_k$}
    \State
    \Let{$\mom^\prime$}{$\mom^\prime +
                \frac{\epsilon}{2}\,\nabla\lnprob(\pos^\prime)$}
        \Comment{run $L$ steps of leapfrog integration}
    \For{$l \gets 1 \textrm{ to } L$}
        \Let{$\pos^\prime$}{$\pos^\prime +
                \epsilon\,{\mass_k}^{-1}\,\mom^\prime$}
        \If{$l < L$}
            \Let{$\mom^\prime$}{$\mom^\prime +
                \epsilon\,\nabla\lnprob(\pos^\prime)$}
        \EndIf
    \EndFor
    \Let{$\mom^\prime$}{$\mom^\prime +
                \frac{\epsilon}{2}\,\nabla\lnprob(\pos^\prime)$}
        \Comment{synchronize the momentum and position}
    \State
    \State{$r \sim \mathcal{U}(0, 1)$}
    \If{$r < \exp\left[
        \lnprob(\pos^\prime)
        - \frac{1}{2}{\mom^\prime}^T{\mass_k}^{-1}\mom^\prime
        - \lnprob(\pos_k)
        + \frac{1}{2}{\mom_k}^T{\mass_k}^{-1}\mom_k \right]$}
        \Let{$\pos_k$}{$\pos^\prime$}   \Comment{accept}
    \Else
        \Let{$\pos_k$}{$\pos_k$}   \Comment{reject}
    \EndIf
\EndFor
\State\Return{$\{ \pos_k \}_{k=1}^K$}
\EndFunction
    \end{algorithmic}
\end{algorithm}

\clearpage
\bibliography{emcee-hmc}
\clearpage

\end{document}
